\subsection{Vacuum Methods}
\subsubsection{Sputtering}
Sputtering: a method of physical vapour deposition, through ejection of a material from a `target' onto a `substrate' such as a silicon wafer. This ejection occurs by running a charge to an anode and cathode (with the cathode formed of the material to be sputtered), and pushing a gas through the sputtering chamber at a high pressure. The sputtered ions may travel ballistically from the target to the substrate, or at higher pressures the gas may act as a moderator and allow the sputtered material to diffuse through the system, condensing and adhering to the substrate after undergoing a random walk.
Often an electromagnet is used to keep plasma particles close to the surface of the sputter target - Rf sputtering is the process of using an RF power source in order to ionize the the gaseous atoms. Once the ions contact the target material, it is broken into small pieces which travel to the substrate, forming a coating.
\begin{figure}
\includegraphics[width=80mm]{sputtering.png}\citep{_sputtering.gif_????}
\end{figure}
An advantage of this method is that metals with high melting points are easily sputtered, whilst evaporation of these materials is problematic. This method causes the film to have a composition very close to that of the source material, with a slight difference due to the spreading of various elements because of their mass differences. These films have a higher adhesion rate than evaporated films, and due to the low temperature they work at, have no hot parts.

This method has achieved a 6.77\% efficiency over a 0.15cm$^2$ active area, representing one of the highest efficiency values for a pure CTZ film. \citep{mitzi_path_2011}

\subsubsection{Evaporation}

The principle deposition method used in many of the early kesterite cells, which was chosen after it's previous successes with CIGSSe materials.
The first functional evaporated kesterite device (power efficiency of 0.66\%, found in 1997) , and subsequent multilayer structures were evaporated at a temperature of 150$^\circ$C, then subject to sulphurization at 500$^\circ$C in 5\% nitrogen in a hydrazine atmosphere.

Later deposition methods substituted ZnS as opposed to Zn, for the bottom layer and an increase in initial temperature to 400$^\circ$C - this lead to 2.62\% efficiency.\citep{todorov_high-efficiency_2010}


\subsection{Non-Vacuum Methods}
\subsubsection{Electrodeposition}
The process of using electrical current to reduce a solution of dissolved metal ions to form a metal coating onto an electrode. This is used to build up several layers of elements onto a substrate to form the thin film solar cell. This process has the side effect of changing the the chemical physical and mechanical properties of the item to be plated, which may have detrimental or unwanted side effects. This is an attractive method for large-scale applications, due to the large number of cells which may be produced simultaneously. Early work in the field began by successively plating each of the metals onto a substrate, before sulfurization with elemental Sulphur. This led to an efficiency of ~0.8\% \citep{scragg_new_2008}, however limitations on device performance included the high series resistance, high shunt conductance and substantial recombination in the space charge region. Other efforts resulted in poor layer adhesion to the substrate, solved through addition of Palladium to the substrate leading to 0.98\% efficiency. The current record efficiency for an electrodeposited CZTS device is 3.4\%, using a Solid state reaction of Cu$_2$SnS$_3$ and ZnS, at 550$^\circ$.
