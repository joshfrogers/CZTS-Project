% Chapter Template

\chapter{Introduction} % Main chapter title

\label{Chapter1} % Change X to a consecutive number; for referencing this chapter elsewhere, use \ref{ChapterX}

\lhead{\emph{Introduction}} % Change X to a consecutive number; this is for the header on each page - perhaps a shortened title

%----------------------------------------------------------------------------------------
%	SECTION 1
%----------------------------------------------------------------------------------------

\section{Main Section 1}

The extent of solar energy reaching the earths surface produces an energy supply potential far exceeding the energy requirement of our planet's needs by three orders of magnitude.\citep{Morton2006} However, this resource has not been harvested to a great value - currently only 0.01\% of worldwide power generation is met by photovoltaic supply. \citep{Mitzi2011} This is largely due to the high cost difference between Solar and more conventional power generation techniques. As such, to gain any real impact upon worldwide power consumption, photovoltaic cells must fit the following characteristics: must be cost effective, highly abundant, provide a supply for a long lifetime before degradation, and have a high power efficiency.


The 1839 discovery of the Photovoltaic effect by Edmund Becquerel \citep{Nelson2003}, whilst experimenting with electrolytic cells, was the first step into the field of solar power and technology, and the first of several discoveries along the path to using the sun as an energy source for power. Following this, in 1883, Charles Fritts\citep{Fritts1883} developed the first solid state solar cell, by coating selenium with a thin layer of gold to form the junctions, this was then followed by a series of successive discoveries between 1839 and 1941 led to the development and patenting of the first "Light Sensitive Device", by Russell Ohl in 1946, a modern junction semiconductor solar cell.\citep{Green2009a} 

This melt grown junction device had less than 1\% efficiency, and was succeeded by a device built by Kingsbury and Ohl produced through Helium-bombardment, boasting an efficiency of ~1\% in 1952.\citep{Green2009a} Further developments in semiconductor photovoltaics led to efficiencies reaching 14\%, but at costs of up to \$250 per watt (compared to \$2-3 for a coal plant) by 1954. Due to the move to integrated circuits within the semiconductor industry, the price of silicon cells dropped to \$100 per watt.

Further improvements to silicon based solar cells have both improved efficiency and reduced the cost, with large arrays able to be built at below \$3.40/watt\citep{Ausick2014}, and as of September 2013, up to 44.7\% efficiency.\citep{Fraunhofer-Gesellschaft2014} Thin film solar cells were originally developed for usage in small scale applications such as calculators, but are now available for usage in much larger scale installations such as car charging systems; produced by sandwiching a measure of photovoltaic material between two panes of glass. Thin Film technologies allow for solar energy to be harvested using much less material than that required by standard semiconductor solar cells, whilst retaining a measure of the efficiency of the larger variety. Due to the reduced need for components, there is a lower environmental impact, but also a much lower efficiency than standard cells.

Strong candidate compounds for Thin Film Solar Cells were Cadmium Telluride (CdTe) and Copper Indium Gallium Selenide (CIGS), showing promise as general purpose cells, and seeing high commercial success, however recent concerns have been voiced over the cost, abundance and toxicity of the materials used in these compounds. The costs of Indium have skyrocketed recently to over \$1000/kg, due its usage in displays, which could limit the maximum possible output from these cells. Tellurium is similarly high priced due to its scarcity being similar to that of Gold. The combinations of these with the Toxicity of Cadmium lead to the requirement of a new Compound, that would fulfil the requirements of a solar cell.

Copper Zinc Tin Sulphide (CZTS) is seen as the compound of choice for solving the problems laid out above, as it is produced from highly abundant, non-toxic materials - leading to a much lower cost.\citep{Wadia2009} CZTS is a chalcopyrite substance, and has been known to geologists since 1958, Kesterite \citep{WebMineral2014}, however was only discovered to have the photovoltaic effect in 1988 \citep{Ito1988}. in 1997, solar cells with an efficiency of 2.3\% were found, and in November 2013 Solar frontier, a Japanese thin-film company developed a CZTS solarcel with 12.6\% efficiency.\citep{Wang2013}

%-----------------------------------
%	SUBSECTION 1
%-----------------------------------
\subsection{test}



%-----------------------------------
%	SUBSECTION 2
%-----------------------------------

\subsection{Subsection 2}

%----------------------------------------------------------------------------------------
%	SECTION 2
%----------------------------------------------------------------------------------------

\section{Theory}

Photovoltaic cells all work of the principle of the Photovoltaic Effect, which can be simplified down to:

	\begin{enumerate}
	\item Photons in sunlight are incident on the solare panel, and are absorbed by a semiconductor.
	\item Electrons are knocked loose from atoms, allowing them to flow through the material, in a single direction (creating a positive terminal and negative terminal)
	\item A large array of these cells can be strung together to produce a DC current.
	\end{enumerate}

	Realistically, the photon can experience one of three possible outcomes: transmission through the surface, reflection off the surface and absorption if the photon's energy exceeds the band gap of the semiconductor. Upon absorption, the photon energy is transferred to a negative electron in the valence band exciting the electron to the conduction band of the material. There it is free to move in the semiconductor, leaving a positive hole where the electron had been. The majority of solar photons have an energy higher than that of the band gap of silicon semiconductors, thus when searching for alternatives to silicon semicondutors we look for alternatives with a similar band gap.

%Intro
%-
%
%History of solar cell tech
%->Production and development of said technology over crystal structure
%Reasons for drive for new technologies
%-> Method of work in solar cells
%-> Reasoning for rejected tech, as such req for new tech
%
%History of CZTS - Diff composition, and difficulty in production
%->Compositional problems, identification - Similar crystal structure
%-> Advances with CZTS, Efficiency, current technologies
%Possible use cases over trad tech.
%
%PYTHON
%------
%
%Trick is not to attempt to assemble 4 ternary diagrams, but to project our current ones along a third dimension 
%to a point - allowing us to take 'snapshot' ternary diagrams that could occur at various concentrations of a given element.
%
%Knowing that we have Certain tie lines beginning and ending along this projection, allows quick pinpointing of where these 'snapshots' need to be.

	
