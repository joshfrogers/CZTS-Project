We begin by determination of the phases stable at a given temperature: initially 298K; These are determined by examining binary alloy phase diagrams for each of the possible element pairs, for example $Cu-Zn$, $Cu-Sn$, $Cu-S$ etc. 

After building a list of possible stable phases, we then produced four Ternary phase diagrams - at the vertices sit three of the four elements, and the edges of the diagram correspond to a varying percentage composition of the elements at either end of the line. 

On these edges, we mark the stable compounds and then draw from each compound a 'tie-line' to the element at the opposite vertex, and to each compound on the remaining edges. 

These tie-lines correspond to stable pairs of phases, and where these lines cross one another, the crossing point represents the a reaction whereby one of the lines represents reactants, and the other the products. In order to determine which are the products, and as such which line should remain, we examine the Gibb's Free Energy of the reactants and by taking the difference, we can see which direction the reaction favours. 

By finding the favoured direction, and as such the favoured tie-line, we can remove the other line, which will subsequently reduce the number of crossing points remaining. We then repeat this system of calculations on remaining crossing points, removing the unfavoured tie-lines until no more crossing points remain, leaving only a set of stable tie-lines. (A more detailed explanation of the calculations will be explained in an Appendix)

We repeat this for each of the possible combinations of elements, producing four Ternary Phase Diagrams, each only having stable tie-lines. Using these, we can then produce a quaternary diagram, by joining each of the Ternary Diagrams and folding to produce a tetrahedron. 

Evident upon the tetrahedron will be the tie-lines from each of the respective Ternary Diagrams, which where they join on three surfaces will produce a 'tie-phase'. These tie-phases will be similar to the Ternary Diagrams, but rather than having elements at the vertices, it will have compounds.

We can look at each tie-plane, to see whether there are any crossing vectors with other tie-planes. If these occur we perform similar calculations to those performed for the tie-lines, and as such remove the unfavoured tie-plane.

We then subsequently either resolve each tie-plane in order to find one which provides the optimum thermodynamic conditions for production of $Cu_2ZnSnS_4$; or choose a tie-plane that has the correct proportions of the required elements, and subsequently resolve it to find the area that $Cu_2ZnSnS_4$ would occur, and the primary and secondary phases associated with it.